\documentclass[oneside,12pt]{memoir}
\def\mychaplineone{Developer notes for}
\def\mychaplinetwo{ESGF Node Manager}
%\def\mychaplinethree{And Some More...}
\def\myauthone{Prashanth Dwarakanath}
\def\myauthtwo{Sasha Ames}
%\def\myauththree{Torgny Fax\'en}
\def\mypress{Earth Sciences Grid Federation}
\usepackage{chengi}
\def\phname{NM{ }}
\def\vernum{December 2, 2014}
\setcounter{tocdepth}{2}
\thispagestyle{empty}
\titleGM
\chapterstyle{BlueBox}
\pagestyle{mystyle}
\begin{document}
\frontmatter
\hypertarget{mytocmarker}
\tableofcontents
\mainmatter
\setcounter{secnumdepth}{2}
\chapter{Solution Design}
\section{Overview}


\section{Design Considerations}

We are interested in a hierarchical framework for node management rather than a pure p2p.  In this case nodes elect "super-nodes" that perform additional tasks than "member-nodes".  Our goal for scalability is to have a balance of super vs member nodes to ensure that no super-node becomes oversubscribed.  Super-nodes operate at the "project" level.   

A node that serves as the "super-node" for project A can be a member node for project B.  Super-nodes can serve in that role for multiple projects.    Configuring a node for eligibility for selection to become a super node is voluntary, but important for the federation to have an "adequate" number of candidates.

Vetting for a super node. Hand-picked initial super-nodes.  

member nodes pass messages.  

Node manager admin interface.  Allow admins to eg. sign certs, vet nodes,  

\begin{enumerate}
\item
Tree-based communication between super nodes and member nodes to reduce communication overheads, ie. avoid $n^2$ patterns.  We expect a "wide" tree with "few" hops. 
\end{enumerate}


\section{Breakup of tasks}



\section{System Architecture}
Diagram of the system to go here...\\

\subsection{Components}
\begin{enumerate}
\item 
super node election service (zookeeper?)
\item
node failure detection (zookeeper?)
\item
shared state among nodes (zookeeper?)
\item
node status service
\item
metrics collection and query

\end{enumerate}

\subsection{Use Case Diagram}
Use case diagram goes here...


\section{Issues and questions}
This is to be a place to list out problems that need to be handled.
\begin{enumerate}
\item
do we use zookeeper?

to what extent can we rely if we do,  what do we need to implement for the nm


Is zookeeper a fit for management?
\item
can zookeeper manage group-level access control?
\item
Deal with "spoofed" node; ensure relayed information is legitimate. 


\end{enumerate}
\hypertarget{mymarker}{}
\printindex
\end{document}
