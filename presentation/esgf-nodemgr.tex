\documentclass{beamer}
\usepackage[latin1]{inputenc}
\usepackage{graphicx}
\usepackage{longtable}
\usepackage{multicol}
\usepackage{biblatex}
\bibliography{gridbib}
\usetheme{Madrid}
\usebackgroundtemplate{
\setlength{\unitlength}{1mm}
\begin{picture}(-10,80)
%\put(5,-10){\includegraphics[scale=0.2]{Logo-v2.pdf}}
\put(95,-10){\includegraphics[scale=0.1]{../logowork/doe-is-enes.pdf}}
\end{picture}
\setlength{\unitlength}{1pt}
}
\title{ESGF Node Manager}
\author{P. Dwarakanath, S. Ames}
\institute{LIU/LLNL}
\date{December 10 2014}
\begin{document}

\begin{frame}
\titlepage
\setlength{\unitlength}{1mm}
\begin{picture}(-10,-10)
%\put(40,-10){\includegraphics[scale=0.3]{Triolith3.jpg}}
\end{picture}
\setlength{\unitlength}{1pt}
\end{frame}

\begin{frame}{Contents}
\tableofcontents
\end{frame}

\section{Background}
\begin{frame}{Background}
The current ESGF node manager handles:
\begin{itemize}
\item Capturing metrics
\item Sharing node information across federations: certs, endpoints etc
\item A mechanism to share common configuration files.
\end{itemize}
Drawbacks
\begin{itemize}
\item Limited scalability.
\item P2P file/data exchange could be more secure, particularly configuration files.
\end{itemize}

\end{frame}

\section{Desirable features for next-gen Node Manager}
\begin{frame}{Desirable features for next-gen Node Manager}
\begin{itemize}
\item Fault-tolerant distributed system, without a single point of failure.
\item High scalability without overloading resources.
\item Minimise communication overheads.
\item PAN federation administration: handling cert requests, node memberships etc
\item Consistent and highly available common configuration files
\end{itemize}
\end{frame}

\section{Overview of proposed solution}
\begin{frame}{Overview of proposed solution}
Supernodes
\begin{itemize}
\item A supernode is a validated and reliable source for configuration directives, metrics, information about components etc.
\item Multiple supernodes to to ensure scalability, fault tolerance and load sharing.
\item Supernodes are designed to function at the Project level.
\item A node can be supernode to multiple projects, or supernode to one/some and member in others.
\item Nodes can be standby supernodes too: only used when another supernode is down.
\end{itemize}
\end{frame}

%\section{Component Diagram for data download workflow}
%\begin{frame}{Component Diagram for data download workflow}
%\includegraphics[scale=0.52]{/home/pchengi/clipc-work/clipc-dev/doc/diagram.pdf}
%\end{frame}

\end{document}
